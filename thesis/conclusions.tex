\chapter{Conclusions and Further Work} \label{chap:concl}

\section*{}
We have developed a framework capable of predicting software defects from
repositories, with a web-based graphical report. The creation of a learning mode
for Schwa with genetic algorithms, gives researchers the ability of evaluating
new features to extract from repositories, making Schwa a convenient framework
to study Mining Software Repositories.

Schwa should be combined with other techniques, since it is not completely
accurate. Code review is an example of an activity that can benefit from this
tool, allowing developers to focus in the most important components.

The usage of Python allowed a fast prototyping of ideas due its simplicity and
the existing of useful libraries. Mining Software Repositories is a
time-consuming activity so research in this subject can benefit from the usage
of clusters.

\section{Goals satisfaction}
We successfully created a defect prediction technique based on MSR approaches
capable of learning features, until the method granularity for Java projects.
Our initial goal of generalizing features weights was refuted by the
experimental results, that shown that for each projects they are different.

Although we did not improve the accuracy of Barinel, we have come with an
alternative technique of computing defect probabilities in less time. For
example, since Barinel for Joda Time can take 2 hours to run MLE, now with
Schwa, this phase takes less that 1 minute, so it is a substantial achievement.

\section{Further work}
The technique used in Schwa for learning features can be improved with
optimizations in the binary representation and code parallelization. There are
plenty of improvements that can be done in Schwa:
\begin{itemize}
\item Support of more programming languages;
\item Improve performance on extraction by developing a Python module in C;
\item Add charts for revisions, fixes and authors evolution in the visualization,
to support the results with more reasoning;
\item Develop a SaaS platform for Schwa, similar to Codeclimate and Codacy.
\end{itemize}

MSR research could benefit of new techniques that reduce noise in the
classification of bug-fixing commits, that can exploit issue trackers. Schwa
could benefit from reducing this noise.

With more computational power, we could evaluate with more examples, the gain of
using Schwa in Crowbar, by finding an example where the diagnostic cost decreased.
